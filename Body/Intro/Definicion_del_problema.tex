\section{Definición del Problema}

El Centro de Innovación y Integración de Tecnologías Avanzadas (CIITA) de Ciudad Juárez enfrenta múltiples obstáculos para alinear sus procesos con los estándares internacionales ISO 9001 e ISO/IEC 17025. Dichos estándares resultan determinantes en el cumplimiento de su misión orientada a la innovación tecnológica en la región. Durante la estancia profesional, se identificaron los siguientes problemas:

\begin{itemize}
	\item \textbf{Gestión Organizacional:}
	\begin{itemize}
		\item \textbf{Ausencia de un Sistema de Gestión de Calidad (SGC):} El CIITA carece de un SGC estructurado, lo que incumple los requisitos de ISO 9001 (cláusula 4.4) y restringe su capacidad para brindar servicios confiables.
	\end{itemize}
	
	\item \textbf{Recursos Humanos:}
	\begin{itemize}
		\item \textbf{Sobrecarga laboral y desconocimiento normativo:} El personal cumple funciones múltiples sin una capacitación adecuada en ISO 9001 e ISO/IEC 17025. Se observó que el 90\% de los colaboradores (12 personas) asume diversos roles, mientras que el 75\% desconoce los requisitos aplicables, lo que retrasa la estandarización de procesos.
	\end{itemize}
	
	\item \textbf{Procesos Técnicos:}
	\begin{itemize}
		\item \textbf{Documentación no contextualizada:} Cerca del 80\% de los manuales operativos analizados replican la estructura de otra sede del CIITA, incluso con referencias geográficas ajenas a Ciudad Juárez. Tal situación vulnera la mejora continua que exige ISO 9001 (cláusula 10.3).
	\end{itemize}
	
	\item \textbf{Difusión y Financiamiento:}
	\begin{itemize}
		\item \textbf{Eliminación del fideicomiso y baja visibilidad regional:} La desaparición de un fondo de inversión (fideicomiso) redujo los recursos destinados al CIITA, generando demoras en la adquisición de materiales y en la ejecución de nuevos proyectos. Además, el Instituto Politécnico Nacional (IPN) no cuenta con un posicionamiento sólido en la región norte del país, lo que dificulta la promoción de servicios y la integración con el sector industrial local.
	\end{itemize}
\end{itemize}

Estos retos han surgido ante la falta de planificación estratégica y la escasa formación del personal en materia de calidad. De no corregirse, se compromete la obtención de certificaciones indispensables para competir a escala nacional e internacional. También se limita la proyección del CIITA como un actor relevante en el desarrollo tecnológico de Ciudad Juárez y su entorno industrial.

\begin{center}
	\includegraphics[width=\linewidth]{../../Pictures/ciita_root_cause}
\end{center}